%!TEX TS-program = pdflatex

\documentclass[12pt]{article}
\usepackage{amsfonts, amsmath, amssymb}
\usepackage{dcolumn, multirow}
\usepackage{setspace}
\usepackage{graphicx}
\usepackage{tabularx}
\usepackage{anysize, indentfirst, setspace}
\usepackage{verbatim, rotating, paralist}
\usepackage{latexsym}
\usepackage{amsthm}
\usepackage{parskip}
\usepackage{hyperref}
\usepackage{color}
\usepackage[right=2.5cm, left=2.5cm, top=3.5cm, bottom=3.5cm]{geometry} %right=, left=, top=, bottom=

\title{Computational Methods Syllabus \\ \emph{Computer Camp}}
\author{Nick Eubank \& Claire Evans}
\date{\today}

\begin{document}
\maketitle


Hello and welcome to Vanderbilt!

\section{Learning Goals}
By the end of Computer Camp, our
\begin{itemize}
    \item Become comfortable with basic functionality of R
    \item Become comfortable with basic functionality of LaTeX
    \item Be aware of tools available for social scientists
\end{itemize}

\section{Why R?}

Because it's currently the most-used statistical software in political science, and using the same tool as those around you will make it easier to collaborate and get help. It is \emph{not} because it is the only tool available, or even the best.

In the course of your careers, you will hear \emph{lots} of opinions about what software is best. Some people swear by R and look down upon people who use other programs like Stata; others hate R and think it's clunky. The reality is that there are currently a large number of programs people use to do statistics in the social sciences, including R, Stata, SPSS, Python, Julia, etc., just to name a few, and \textbf{each language has its own strengths and weaknesses} and over the course of your career you may find yourself gravitating to one tool or another as required by your research. But most programming languages have similar underlying patterns, and by becoming comfortable with R, we hope to not only give you a firm foundation in a very popular language, but also teach you skills you can bring with you if you ever find you need to switch programs.

\section{Class Organization}

In this class we will be ``flipping the classroom'' -- you will be required to review tutorials on R between classes, and each afternoon we will focus on doing exercises that allow you to get hands on experience with the skills you've read about in an environment where help will be available.

In particular, we will primarily relying on a set of excellent R tutorials written by Simon Edjmeyr which you can find at:
\begin{itemize}
    \item \href{https://sejdemyr.github.io/r-tutorials/basics/}{https://sejdemyr.github.io/r-tutorials/basics/}
\end{itemize}
as well as a few supplementary tutorials from Kelly Black which you can find at:
\begin{itemize}
    \item \href{https://www.cyclismo.org/tutorial/R/}{https://www.cyclismo.org/tutorial/R/}
\end{itemize}


\section{Schedule}


\subsection*{Class 1: Getting Started}
\emph{Thursday, August 9th}
\begin{itemize}
    \item Discuss course goals
    \item Do baseline knowledge assessment
    \item Install R on personal laptops
    \item Look at tutorial pages!
\end{itemize}

\textbf{Homework to be done before Class 2:}
\begin{itemize}
    \item Edjemyr Tutorial: \href{https://sejdemyr.github.io/r-tutorials/basics/introduction/}{Intro to R}
    \item Edjemyr Tutorial: \href{https://sejdemyr.github.io/r-tutorials/basics/vectors/}{Vectors}
    \item Edjemyr Tutorial: \href{https://sejdemyr.github.io/r-tutorials/basics/dataset-basics/}{DataFrames}
\end{itemize}

\emph{Don't worry! This will be BY FAR the biggest set of homeworks. Other nights will be better, but it's hard to do much until we've covered all this.}


\subsection*{Class 2: Basic R Manipulations}
\emph{Friday, August 10th}
\begin{itemize}
    \item read in CSV
    \item look at data, play with couple data types, get summary stats.
\end{itemize}

\textbf{Homework to be done before Class 3:}
\begin{itemize}
    \item Edjemyr Tutorial: \href{https://sejdemyr.github.io/r-tutorials/basics/merging-appending/}{Merging}
    \item Black Tutorial: \href{https://www.cyclismo.org/tutorial/R/linearLeastSquares.html}{OLS}
\end{itemize}



\subsection*{Class 3: Working with Multiple Data Sources}
\emph{Monday, August 13th}
\begin{itemize}
    \item read in CSV
    \item look at data, play with couple data types, get summary stats.
\end{itemize}

\textbf{Homework to be done before Class 3:}
\begin{itemize}
    \item Edjmeyr Tutorial: \href{https://sejdemyr.github.io/r-tutorials/statistics/tutorial2.html#graphing-with-ggplot}{ggplot}
    \item Edjmeyr Tutorial: \href{https://sejdemyr.github.io/r-tutorials/basics/modifying-data/}{dplyr}
\end{itemize}


\subsection*{Class 4: Ad Hoc Day}
\emph{Tuesday, August 14th}
\begin{itemize}
    \item ?
\end{itemize}

\textbf{\emph{Wednesday, August 15th}: NO CLASS}

\subsection*{Class 5: LaTeX}
\emph{Thursday, August 16th}
\begin{itemize}
    \item Teach latex!
\end{itemize}

\subsection*{Class 5: Tools You'll Want in Profession}
\emph{Friday, August 17th}
\begin{itemize}
    \item Wolfram Alpha
    \item Mendeley / Papers
\end{itemize}


\end{document}
